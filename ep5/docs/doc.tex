\documentclass[12pt]{article}

\usepackage[brazilian]{babel}
\usepackage[utf8]{inputenc}
\usepackage{graphicx}
\usepackage{mathtools}
\usepackage{amsthm}
\usepackage{thmtools,thm-restate}
\usepackage{amsfonts}
\usepackage{hyperref}
\usepackage[singlelinecheck=false]{caption}
\usepackage{enumitem}
\usepackage[justification=centering]{caption}
\usepackage{indentfirst}
\usepackage{algorithm}
\usepackage{algpseudocode}
\usepackage{listings}
\usepackage[x11names,rgb,table]{xcolor}
\usepackage{tikz}
\usepackage{hyperref}
\usepackage{subcaption}
\usepackage{booktabs}
\usepackage{linegoal}
\usepackage{geometry}
\usepackage{interval}
\usepackage[normalem]{ulem}
\usetikzlibrary{snakes,arrows,shapes}

\graphicspath{{imgs/}}

\makeatletter
\def\subsection{\@startsection{subsection}{3}%
  \z@{.5\linespacing\@plus.7\linespacing}{.1\linespacing}%
  {\normalfont}}
\makeatother

\DeclareMathOperator*{\argmin}{arg\,min}
\DeclareMathOperator*{\argmax}{arg\,max}
\DeclareMathOperator*{\Val}{\text{Val}}
\DeclareMathOperator*{\Ch}{\text{Ch}}
\DeclareMathOperator*{\Pa}{\text{Pa}}
\DeclareMathOperator*{\Sc}{\text{Sc}}
\newcommand{\ov}{\overline}
\newcommand{\tsup}{\textsuperscript}

\newcommand\defeq{\mathrel{\overset{\makebox[0pt]{\mbox{\normalfont\tiny\sffamily def}}}{=}}}

\newcommand{\algorithmautorefname}{Algorithm}
\algrenewcommand\algorithmicrequire{\textbf{Input}}
\algrenewcommand\algorithmicensure{\textbf{Output}}
\algnewcommand{\LineComment}[1]{\State\,\(\triangleright\) #1}

\captionsetup[table]{labelsep=space}

\theoremstyle{plain}

\newcounter{dummy-def}\numberwithin{dummy-def}{section}
\newtheorem{definition}[dummy-def]{Definition}
\newcounter{dummy-thm}\numberwithin{dummy-thm}{section}
\newtheorem{theorem}[dummy-thm]{Theorem}
\newcounter{dummy-prop}\numberwithin{dummy-prop}{section}
\newtheorem{proposition}[dummy-prop]{Proposition}
\newcounter{dummy-corollary}\numberwithin{dummy-corollary}{section}
\newtheorem{corollary}[dummy-corollary]{Corollary}
\newcounter{dummy-lemma}\numberwithin{dummy-lemma}{section}
\newtheorem{lemma}[dummy-lemma]{Lemma}
\newcounter{dummy-ex}\numberwithin{dummy-ex}{section}
\newtheorem{exercise}[dummy-ex]{Exercise}
\newcounter{dummy-eg}\numberwithin{dummy-eg}{section}
\newtheorem{example}[dummy-eg]{Example}

\numberwithin{equation}{section}

\newcommand{\set}[1]{\mathbf{#1}}
\newcommand{\pr}{\mathbb{P}}
\newcommand{\eps}{\varepsilon}
\newcommand{\ddspn}[2]{\frac{\partial#1}{\partial#2}}
\newcommand{\iddspn}[2]{\partial#1/\partial#2}
\renewcommand{\implies}{\Rightarrow}

\newcommand{\bigo}{\mathcal{O}}

\setlength{\parskip}{1em}

\lstset{frameround=fttt,
  frame=single,
  keepspaces=true,
	breaklines=true,
	keywordstyle=\bfseries,
	basicstyle=\footnotesize\ttfamily,
  language=Python
}

\newcommand{\code}[1]{\lstinline[mathescape=true]{#1}}
\newcommand{\mcode}[1]{\lstinline[mathescape]!#1!}

\newgeometry{margin=1in}
\title{%
  \textbf{Relatório MAC0318 EP5}\\
  \author{Renato Lui Geh --- NUSP\@: 8536030}
}
\date{}

\begin{document}

\maketitle

\section{Definições e notação}

Vamos chamar de $X$ a variável aleatória discreta e finita que representa em que pedaço do espaço
unidimensional discretizado o robô está. Então $P(X=x)$ é a probabilidade do robô estar na
``célula'' $x$. Vou chamar de $m$ o número de células total. Suponha que o espaço unidimensional
mede $d$. Vou chamar de $\ov{d}$ o tamanho de cada célula, ou seja, $\ov{d}=\lfloor d/m\rfloor$.

A variável aleatória discreta e finita $Z$ representa o sonar. Como dito no enunciado,
$Z\sim\mathcal{N}$. As gaussianas representando os erros e ruídos do sonar quando captando as
caixas e buracos serão denotadas por $\mathcal{N}_b$ e $\mathcal{N}_g$ respectivamente.

Na probabilidade de ação $P(X'=x'|X=x,u)$, $X'$ é a variável aleatória da posição do robô na célula
$x'$ após a ação $u$.

\section{Execução}

Para rodar o EP5, é só rodar com \code{python3}. Se o script for rodado sem argumentos, então o
código esperará que o robô Lejos esteja conectado por cabo USB\@. Para rodar apenas a simulação, rode
com qualquer um dos argumentos abaixo:\\

\begin{lstlisting}
python3 localization.py -s
python3 localization.py --simulate
\end{lstlisting}

Assim que o código terminar de pré-computar as matrizes de probabilidade, o programa vai abrir uma
janela com um histograma da distribuição de probabilidade $P(X)$. Neste histograma, as barras
vermelhas e amarelas são as probabilidades do robô estar nas respectivas células. Quando uma barra
aparece em vermelho, é por que existe uma caixa na célula. Quando a barra aparece amarela, a célula
não possue caixa. O fundo é colorido de acordo com as caixas e buracos, onde laranja indica caixa e
verde vazio. Uma barra azul indica a posição real do robô durante a simulação.

Após aberto o histograma, o programa continuará rodando a espera de \textit{input}. A tela mostrará
algo semelhante a:\\

\begin{lstlisting}
Starting simulation at position: 5
Ready. Press ? for help message.
\end{lstlisting}

Os comandos para o programa funcionam igual ao \texttt{NORMAL mode} do Vim. Um comando é
representado por um caractere. O caractere pode conter um quantificador (um número) que altera o
comportamento do comando. Abaixo segue a lista de comandos dada pelo comando de ajuda \code{?}:\\

\begin{lstlisting}
-------------------------
This controller works very much like vim. Available commands are:
  h - Go left and apply correction and prediction.
  j - No-op. Dont move, but apply correction and prediction.
  l - Go right and apply correction and prediction.
  k - Force correction only, with no prediction.
  c - Shows the current models constraints and localization settings.
  q - Quit.
  ? - Show this help message
Capitalized equivalents apply only prediction with no correction:
  H - Go left and apply only prediction.
  J - No-op. Dont move, but apply prediction only.
  K - No-op. Applies correction. The same as k.
  L - Go right and apply only prediction.
Every command can be quantified (just like vim!). A number before a command means the command should be repeated that many times. For example:
  2l  - Go right two units and then apply correction and prediction.
  10H - Go left ten units and then apply only prediction.
  j   - Compute prediction and correction values and dont move.
  5k  - Compute correction values five times.
When omitting a quantifier, the command assumes the quantifier is 1.
-------------------------
\end{lstlisting}

Os comandos são lidos em tempo-real, então logo que o programa receber um comando válido, ele o
executará. Assim não é preciso ficar apertando Enter para validar o input. Se o comando não for
válido, o programa descartará o input anterior e pedirá uma nova sequência de caracteres para
input.

Quando aberto, o histograma é posto em foco. Esta será a única vez que a janela será dada
foco. É possível manter a janela aberta de lado enquanto são dados os comandos que isso não fará
com que o foco seja alterado. Em vários Desktop Environments, colocar a janela em ``Always on top''
pode facilitar a vida.

Como a mensagem de ajuda mostra, o quantificador modifica o comportamento do comando. Omitir este
quantificador é a mesma coisa que se o quantificador fosse 1.

Após cada comando de movimento ou computação, uma mensagem indicará qual comando foi feito e quais
são as posições reais do robô antes e depois do movimento.

Se o robô fosse sair do intervalo $\interval[open right]{0}{m}$, o robô permanece nos bounds (0 ou
$m$).

O comando \code{c} imprime informações úteis do mapa, gaussianas e do \code{Range} (que será
comentado posteriormente). \sout{O robô não pode andar mais do que o que estiver determinado em ``Bounds
for number of steps'' de uma só vez}. Como agora a matriz de probabilidade de ação não é mais
pré-computada, e ao invés disso as probabilidades são computadas durante a movimentação, o robô não
tem mais limite de passos em uma iteração. Por exemplo:\\

\begin{lstlisting}
Map Properties:
  Sensor Gaussians:
    Means: 10.0 | Variance: 1.0
    Means: 15.0 | Variance: 1.0
  Discretization bin size: 100
  True size of each bin: 1.7
  Size of precomputed probability matrices:
    P(Z|X)    (sensor probability distribution): (20, 100)
  Bot attached? False
Range Properties:
  Unique commands available: [-1, 0, 1]
  Bounds for number of steps: [-25, 0, 25]
  Pivot: 24
\end{lstlisting}

\section{Configurações}

Para mudar a configuração do espaço, das gaussianas, matrizes de probabilidade ou discretização, é
preciso mudar o próprio código. Para isso, foram criadas algumas funções.

A função \code{new_config} cria uma nova configuração de mapa. Ela toma como argumentos (em ordem):

\begin{enumerate}
  \item $\mu_b$: Média da gaussiana do sonar para as caixas.
  \item $\sigma_b$: Variância da gaussiana do sonar para as caixas.
  \item $\mu_g$: Média da gaussiana do sonar para os buracos.
  \item $\sigma_g$: Variância da gaussiana do sonar para os buracos.
  \item $C$: Vetor de distâncias.
  \item \mcode{starts_with$\in\{$\"gap\"$, $\"box\"$\}$}: Estado inicial.
  \item $b$: Número de bins para discretização.
\end{enumerate}

As gaussianas $\mathcal{N}_b(\mu_b, \sigma_b)$ e $\mathcal{N}_g(\mu_g, \sigma_g)$ são as
distribuições de probabilidade para os erros e ruídos do sonar.

O vetor de distâncias $C$ é um vetor onde cada entrada $i$ representa a distância do $i$-ésimo
objeto (caixa ou buraco) até o $i+1$-ésimo objeto. Junto com \code{starts_with}, que indica se $C$
começa com uma caixa ou buraco, $C$ representa o ambiente unidimensional inteiro.

O argumento $b$ é o número de bins para discretização. Supondo que $C$ é dado em centímetros, então
se $b=\sum_{i=0}^{|C|-1} C_i$, então cada bin terá tamanho de 1 centímetro.

A função retorna três objetos:

\begin{enumerate}
  \item $M$: Mapa de tamanho $b$, com $M_i=0$ se a célula $i$ é buraco e $M_i=1$ se é caixa.
  \item $(\mathcal{N}_g,\mathcal{N}_b)$: Par ordenado com as gaussianas relevantes.
  \item $\ov{d}$: Tamanho de cada célula na mesma unidade dada em $C$.
\end{enumerate}

A função \code{new_config} então cria um mapa do espaço discretizado e cria as gaussianas para o
sensor. No código original temos:\\

\begin{lstlisting}
C, N, d = new_config(15, 1, 10, 1, [5, 10, 15, 10, 20, 10, 30, 10, 15, 10, 20, 10, 5], bin_size=100)
\end{lstlisting}

Após criado o mapa do espaço, devemos criar a distribuição de probabilidade $P(X)$ inicial. Para
isso, usamos a função \code{gen_init_pdist}, que toma como argumentos:

\begin{enumerate}
  \item $n$: Número de valores possíveis para $X$, ou seja, número total de células.
  \item $\mu$: Média para gaussiana.
  \item $\sigma$: Variância para gaussiana.
  \item $s$: Tamanho de amostras para gerar a distribuição.
  \item $u$: Booleana que indica se a distribuição é uniforme.
\end{enumerate}

A função gera ou uma uniforme de $\mathcal{U}(0, n)$ ou uma gaussiana $\mathcal{N}(\mu,\sigma)$. Se
$u$ for \code{True}, então ignora-se todos os argumentos menos $n$. Caso contrário, serão geradas
$s$ amostras de uma gaussiana $\mathcal{N}(\mu,\sigma)$ para formar a distribuição inicial.

O valor de retorno da função é um vetor $P$ distribuição de probabilidade com $n$ entradas. No
código original temos:\\

\begin{lstlisting}
# Uniform initial belief.
U = gen_init_pdist(len(C), uniform=True)
# Gaussian initial belief centered on second box.
G1 = gen_init_pdist(len(C), mu=20, sigma=math.sqrt(40))
# Gaussian initial belief centered on fourth box.
G2 = gen_init_pdist(len(C), mu=60, sigma=math.sqrt(40))
\end{lstlisting}

Finalmente, criamos um objeto da classe \code{Map}. O construtor toma os valores de retorno das
funções acima e mais dois argumentos.

\begin{enumerate}
  \item $C$: Mapa das células.
  \item $N$: Par de gaussianas dos sensores.
  \item $d$: Tamanho real de cada célula.
  \item $P$: Distribuição de probabilidade inicial $P(X)$.
  \item $p$: Precisão das matrizes pré-computadas.
  \item $R$: Objeto da classe \code{Range}.
\end{enumerate}

Quando pré-computamos as matrizes de probabilidade, consideramos a área das gaussianas entre
$[-p\sigma,+p\sigma]$, onde $\sigma$ é o desvio padrão. Então com $p=3$, cobrimos 99.7\% da
distribuição.

A classe \code{Range} representa quais são os possíveis movimentos do robô. \sout{Além disso, também
limita o número de passos pré-computados e define a variância da gaussiana para a probabilidade de
ação $P(X'=x'|X=x,u)$}. Como agora a matriz $P(X'=x'|X=x,u)$ não é mais pré-computada, Range apenas
serve para armazenar os possíveis comandos. O construtor de Range recebe:

\begin{enumerate}
  \item $M$: Possíveis movimentações.
  \item $B$: Bounds para movimentação (se a matriz não é pré-computada, não se usa $B$).
  \item $\sigma$: Percentagem da variância para gaussiana da probabilidade de ação.
\end{enumerate}

Portanto, temos no código original:\\

\begin{lstlisting}
M = Map(C, N, d, G1, 3, Range([-1, 0, 1], [-25, 0, 25], 0.5))
\end{lstlisting}

O último argumento modificável é a posição inicial do robô. A função \code{start} tem como terceiro
argumento um número no intervalo $\interval[open right]{0}{m}$, onde $m$ é o número total de
células. Este número indica onde o robô começará.

\section{Observações}

Quando o robô se move por um comando, o simulator decide a movimentação por meio da gaussiana
$\mathcal{N}(\mu=x+d,\sigma=pd)$, onde $x$ é a célula em que o robô está e $d$ é o quanto que o
usuário deseja que robô ande. A variância $\sigma$ é dada pela variância descrita na classe
\code{Range} multiplicado a distância (e.g.\ se $p=0.1$, então a variância será 10\% da distância a
se percorrer). A movimentação real é uma amostragem desta gaussiana, e portanto aleatória. A
linha em que se computa essa amostragem:\\

\begin{lstlisting}
mu = self.pos+u*d
std = abs(u*d*self.R.p)
dp = int(round(stats.norm(mu, math.sqrt(std)).rvs()))
self.pos = min(max(0, dp), self.m-1) 
\end{lstlisting}

Quando $p$ é muito baixo ($\approx 10\%$), as probabilidades ficam muito degeneradas (o modelo
supõe com muita certeza --- ou talvez pouca incerteza --- onde ele supostamente deveria estar).
Quando $p$ é razoavelmente alto ($\approx 50\%$), as probabilidades ficam mais incertas, com várias
seções do histograma parecendo mais com gaussianas. O meio termo ($\approx 25\%$) é o caso mais
razoável.

\end{document}
